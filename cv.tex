\documentclass[a4paper,twoside]{simplecv}
\usepackage[utf8]{inputenc}
\usepackage[hidelinks]{hyperref}
\usepackage{multicol}

\begin{document}

\leftheader{Avenida Central Bloco 1645\\
Lote 10, ap 102\\
Núcleo Bandeirante\\
Brasília, DF\\
71710-560 \\
Brasil}

\rightheader{\href{tel:+5561999894025}{+55 61 99989-4025}\\
\href{mailto:rogi@skylittlesystem.org}{rogi@skylittlesystem.org}\\
\url{https://github.com/ayghor}}

\title{Igor de Sant'Ana Fontana}

\maketitle

Programador autodidata e estudante de matemática. Desenvolvedor backend com 5
anos de experiência.

\section{Experiência}

\begin{topic}
\item[2017] Desenvolvedor Ruby Backend\\
	{\em\small Wololo--We Convert}

	Modelagem de sistema de informação e desenvolvimento de API JSON, em
	Ruby, com Sequel e Roda

\item[2016--2017] Desenvolvedor Rails\\
	{\em\small Papo de Homem}

	Manutenção e desenvolvimento do blog \emph{Papo de Homem}, em Rails.

\item[2013--2014] Desenvolvedor Rails e Java\\
	{\em\small LambLamb--Impressões Impressionantes}

	Manutenção e desenvolvimento de sistema de impressão instantânea
	integrado ao \emph{Instagram}, em Rails e Java.

\item[2011--2013] Desenvolvedor Júnior\\
	{\em\small Bebop--Computação Criativa}

	Assistencia no desenvolvimento de soluções de computação física,
	interatividade e robótica para o mercado de publicidade, entretenimento
	e eventos.
\end{topic}

\section{Formação Acadêmica}

\begin{topic}
\item[2009--2019] Bacharelado em Matemática\\
	{\em\small UnB--Universidade de Brasília}

\end{topic}

\section{Certificados}

\begin{topic}
\item[2016] Machine Learning\\
	{\em\small Coursera--Stanford}

	{\scriptsize\url{https://www.coursera.org/account/accomplishments/verify/NE32SCRBV3F6}}
\end{topic}

\section{Cursos}

\begin{topic}
\item[2017] Bitcoin and Cryptocurrency Technologies\\
	{\em\small Coursera--Princeton}
\item[2013] Rails para Designers\\
	{\em\small Julio Protzek}
\item[2012] Microcontrolador MSP430 com memória FRAM\\
	{\em\small TECHtraininG}
\item[2011] Arduino Hack Day\\
	{\em\small Mecajun--IEEE}
\end{topic}

\section{Projetos}

\begin{topic}
\item[2006--Presente] Motor de jogo 3D

	Motor de jogo 3D inspirada em Quake, escrita do zero em C. Ainda está
	em desenvolvimento, mas possui um demo funcional com seu próprio
	compilador BSP.

	{\scriptsize\url{https://github.com/ayghor/engine-on-Qua-Ago-29-09\_42\_48-BRT-2007}}

\item[2010--2011] Pesquisa sobre métricas de vídeo sem referência\\
	{\em\small UnB--Mylène Farias}

	Pesquisa sobre métricas de vídeo sem referência coordenada pela
	professora \emph{Mylène Farias}.

\item[2006--2011] Linux From Scratch

	Desenvolvi uma distribuição Linux baseada em Slackware e no livro Linux
	From Scratch, que usei por mais de 4 anos.
\end{topic}

\section{Tecnologias}

\begin{multicols}{4}
	\raggedcolumns
	\begin{itemize}
		\item Linux
		\item C
		\item Ruby
		\item Vim
		\item Git
		\item Rails
		\item \LaTeX{}
		\item PostgreSQL
		\item OpenGL
		\item AWS
		\item Heroku
		\item Java
		\item ReactJS
		\item Redux
		\item Bitcoin
		\item Octave
		\item Machine Learning
	\end{itemize}
\end{multicols}

\section{Idiomas}

\begin{multicols}{4}
	\raggedcolumns
	\begin{itemize}
		\item Português
		\item Inglês
	\end{itemize}
\end{multicols}

%\section{Certificados}
%
%\begin{thebibliography}{1}
%\bibitem{mlcert} {}
%\end{thebibliography}

\end{document}
