\documentclass[a4paper,10pt]{moderncv}
\usepackage[utf8]{inputenc}
\usepackage[inline]{enumitem}
\setlist{noitemsep}

\moderncvstyle{casual}
\moderncvcolor{red}
\usepackage[scale=0.75]{geometry}

\name{Igor}{Fontana}
\title{Developer}
\address{88D Grove Street}{SE8 3AA, London}{United Kingdom}
\phone[mobile]{+44~7376~026007}
\email{rogi@skylittlesystem.org}
\social[github]{ayghor}
\social[linkedin]{igor-sf}
\photo[64pt][0.4pt]{pudim_sqr}
\quote{0x5f3759df}

\newcommand*{\mainSkillsItems}{%
	\item Linux
	\item C
	\item Ruby on Rails
	\item SQL
	\item Vim
	\item Git
}

\newcommand*{\othersSkillsItems}{%
	\item AWS
	\item Heroku
	\item Java
	\item OpenGL
	\item libSDL
	\item Octave
	\item Machine Learning
	\item Cryptocurrencies
	\item \LaTeX{}
}

\usepackage[brazilian]{babel}
\title{Desenvolvedor Fullstack}

\begin{document}
\makecvtitle

\section{Sumário}
\cvitem{}{%
	Sou um programador autodidata e estudante de matemática que ama
	soluções elegantes para problemas reais. Uso \emph{Linux} e programo em
	\emph{C} desde os 12 anos ($\approx 17$ anos atrás) e tenho trabalhado
	com \emph{Ruby} e \emph{Rails} nos últimos 5 anos. Acredito que
	tecnologia deve ser simples e útil. Me interesso por \emph{Software
	Livre}, \emph{Videogames}, \emph{Machine Learning}, ensinar e tocar
	violão.
}

\section{Experiência}
\cventry{2017}{Desenvolvedor Backend}{Wololo}{Recife}{}{
	Plataforma de publicidade e gestão de militantes.
	\begin{itemize}
		\item Levantei requisitos a partir de esboços e explicações dos
			designers de produto
%
		\item Desenvolvi um backend rápido e leve utilizando gems
			minimalistas e SQL otimizado
%
		\item Escrevi documentação detalhada para os desenvolvedores
			frontend
	\end{itemize}
}

\cventry{2016--2017}{Desenvolvedor Fullstack}{Papo de Homem}{Remoto}{}{
	Blog sobre masculinidade e assuntos relacionados.
	\begin{itemize}
		\item Corrigi grande quantidade de bugs
%
		\item Mantive o blog no ar 24h, resolvendo problemas
			rapidamente
%
		\item Implementei SSL, melhorando a colocação em motores de
			busca
%
		\item Implementei uma API JSON e notificações push, permitindo
			a criação dos aplicativos mobile
	\end{itemize}
}

\cventry{2013--2014}{Desenvolvedor Fullstack}{LambLamb}{Brasília}{}{
	Impressora instantânea de fotos do Instagram para eventos.
	\begin{itemize}
		\item Melhorei a qualidade das impressões utilizando gráficos
			vetorizados e filtros de redimensionamento adequados
%
		\item Criei editor WYSIWYG de templates de impressão, para
			fácil personalização com marcas dos clientes e
			metadados das mídias
%
		\item Operei e apresentei a LambLamb ao público durante
			eventos
	\end{itemize}
}

\cventry{2011--2013}{Desenvolvedor Júnior}{Bebop}{Brasília}{}{
	Soluções em robótica e computação física para o mercado de publicidade
	e entretenimento.
	\begin{itemize}
		\item Auxiliei no desenvolvimento de instalações interativas
			utilizando visão computacional e projeção mapeada
%
		\item Melhorei o desempenho, especialmente gráfico, de diversos
			produtos da empresa
%
		\item Modelei matematicamente problemas físicos
%
		\item Operei e apresentei as criações ao público durante
			eventos
	\end{itemize}
}

\section{Habilidades}

\cvitem{Main}{%
\begin{itemize*}
	\mainSkillsItems
\end{itemize*}%
}

\cvitem{Others}{%
\begin{itemize*}
	\othersSkillsItems
\end{itemize*}%
}

\section{Idiomas}
\cvitemwithcomment{Português}{Nativo}{}
\cvitemwithcomment{Inglês}{Avançado}{}

\section{Formação Acadêmica}

\cventry{2019}{Bacharelado em Matemática}{Universidade de Brasília}{}{}{}

\section{Cursos}
% FIXME: Princeton
\cventry{2017}{Bitcoin and Cryptocurrency Technologies}{Coursera}{}{}{}
% FIXME: Stanford
\cventry{2016}{Machine Learning}{Coursera}{}{}{
	{\scriptsize\url{https://www.coursera.org/account/accomplishments/verify/NE32SCRBV3F6}}
}
\cventry{2015}{Learning How to Learn}{Coursera}{}{}{}
\cventry{2012}{MSP430 com memória FRAM}{TECHtraininG}{}{}{}
\cventry{2011}{Arduino Hack Day}{IEEE}{}{}{}

\section{Projetos}
\cventry{2017--Agora}{Intro da Raspberry~PI à comunidade do Igapó-Açu}{}{}{}{}

\cventry{2006--Agora}{Teh Engine -- Motor de jogos 3D}{}{}{}{
	{\scriptsize\url{https://github.com/ayghor/teh\_engine}}
}

\cventry{2006--2011}{Rogix -- Distribuição Linux}{}{}{}{
	{\scriptsize\url{https://github.com/ayghor/rogix\_gordax}}
}

\end{document}
