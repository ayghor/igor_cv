\documentclass[a4paper]{simplecv}
\usepackage[utf8]{inputenc}
\usepackage[brazilian]{babel}
\usepackage[hidelinks]{hyperref}
\usepackage{multicol}

\begin{document}

\leftheader{Avenida Central Bloco 1645\\
Lote 10, ap 102\\
Núcleo Bandeirante\\
Brasília, DF\\
71710-560 \\
Brasil}

\rightheader{\href{tel:+5561999894025}{+55 61 99989-4025}\\
\href{mailto:rogi@skylittlesystem.org}{rogi@skylittlesystem.org}\\
\url{https://github.com/ayghor}}

\title{Igor de Sant'Ana Fontana}

\maketitle

Sou um programador autodidata e estudante de matemática que ama soluções
elegantes para problemas reais. Programo em \emph{C} desde os 12 anos ($\approx
17$ anos atrás) e tenho trabalhado com \emph{Ruby} e \emph{Rails} nos últimos 5
anos.

Acredito que tecnologia deve ser simples e útil. Gosto de trabalhar com
\emph{Linux} e \emph{Software Livre} no geral. Também me interesso por
\emph{Videogames}, \emph{Machine Learning}, ensinar e tocar violão.

\section{Experiência}

\begin{topic}
\item[2017] Desenvolvedor Ruby Backend\\
	{\em\small Wololo--We Convert}

	Encarregado de modelar e implementar o backend de uma plataforma de
	publicidade e gestão de militantes. Trabalhei paralelamente com
	desenvolvedores frontend e designers.

	Modelei o banco de dados e a lógica de negócios da aplicação de acordo
	com esboços dos designers de produto. Implementei uma API JSON
	utilizando a bibliotecas minimalistas para garantir a leveza e rapidez
	do servidor. Escrevi documentação detalhada para os desenvolvedores
	web, iOS e Android. Configurei servidores de teste e produção.

	{\em\scriptsize Tecnologias: Ruby, Sequel (ruby gem), Roda (ruby gem),
	PostgreSQL, AWS (Amazon Web Services), EC2 (Elastic Compute Cloud), SES
	(Simple Email Service), SNS (Simple Notification Service), FCM
	(Firebase Cloud Messaging), Facebook Graph API, Facebook Login.}

\item[2016--2017] Desenvolvedor Ruby on Rails\\
	{\em\small Papo de Homem}

	Responsável por resolver problemas diversos e implementar novas
	funcionalidades tanto no backend quanto no frontend do blog \emph{Papo
	de Homem}.

	Corrigi muitos bugs. Migrei banco de dados antigo. Implementei SSL.
	Criei uma API JSON e implementei notificações Push para integração com
	os apps mobile.

	{\em\scriptsize \url{https://papodehomem.com.br}}

	{\em\scriptsize Tecnologias: Ruby on Rails, HTML, Slim (ruby gem),
	JavaScript, LetsEncrypt, Heroku, PostgreSQL, FCM (Firebase Cloud
	Messaging), Elasticsearch, Redis, SideKiq, Facebook Social Plugins.}

\item[2013--2014] Desenvolvedor Rails e Java\\
	{\em\small LambLamb--Impressões Impressionantes}

	Responsável por manter, desenvolver, operar e dar suporte a outros
	operadores da impressora de fotos do \emph{Instagram} --
	\emph{LambLamb}. Trabalhei em dupla com outro programador.

	Melhorei a qualidade das imagens impressas substituindo o formato de
	imagem rasterizado por vetorizado. Criei um editor WYSIWYG de templates
	de impressão, para personalização com logomarcas de clientes e
	metadados da mídia. Operei e apresentei a \emph{LambLamb} ao público
	durante eventos.

	{\em\scriptsize Tecnologias: Ruby on Rails, Java, Java Printing API,
	Java 2D Graphics API, Swing, HTML, JavaScript, Instagram API.}

\item[2011--2013] Desenvolvedor Júnior\\
	{\em\small Bebop--Computação Criativa}

	Assistente no desenvolvimento de objetos e instalações interativas.

	Melhorei o desempenho, especialmente gráfico, de diversos projetos da
	empresa. Modelei matematicamente problemas físicos. Operei e apresentei
	as criações ao público durante eventos.

	{\em\scriptsize Tecnologias: C, Java, Processing, Arduino, XBee,
	Kinect, OpenCV, OpenGL.}
\end{topic}

\section{Formação Acadêmica}

\begin{topic}
\item[2009--2019] Bacharelado em Matemática\\
	{\em\small UnB--Universidade de Brasília}

\end{topic}

\section{Certificados}

\begin{topic}
\item[2016] Machine Learning\\
	{\em\small Coursera--Stanford}

	Ampla introdução sobre aprendizado de máquina, mineração de dados e
	reconhecimento de padrões estatísticos.

	Tópicos incluem:
	\begin{enumerate}
		\item Aprendizagem supervisionada (algoritmos paramétricos e
			não-paramétricos, support vector machines, núcleos,
			redes neurais)

		\item Aprendizagem não supervisionada (agrupamento, redução de
			dimensionalidade, sistemas de recomendação, deep
			learning)

		\item Melhores práticas em aprendizado de máquina (teoria de
			viés/variância; processo de inovação em aprendizado de
			máquina e IA)
	\end{enumerate}

	Ministrado pelo professor \emph{Andrew Ng} da \emph{Universidade de
	Stanford} através da plataforma online \emph{Coursera}.

	{\scriptsize\url{https://www.coursera.org/account/accomplishments/verify/NE32SCRBV3F6}}

	{\em\scriptsize Tecnologias: Octave.}
\end{topic}

\section{Cursos}

\begin{topic}
\item[2017] Bitcoin and Cryptocurrency Technologies\\
	{\em\small Coursera--Princeton}

	Introdução aos aspectos técnicos, economicos e sociais da moeda
	criptográfica Bitcoin.

	Tópicos incluem:
	\begin{enumerate}
		\item Criptografia (simétrica e assimétrica, assinatura
			digital, hash, hash pointer, block chain, merkle tree)
		\item Descentralização da Bitcoin (problema dos generais
			bizantinos)
		\item Comunidade, política e regulação
		\item Bitcoin como plataforma
		\item Altcoins
	\end{enumerate}

	Ministrado pelo professor \emph{Arvind Narayanan} da
	\emph{Universidade de Princeton} através da plataforma online
	\emph{Coursera}.

	{\em\scriptsize Tecnologias: Java.}

\item[2013] Rails para Designers\\
	{\em\small Julio Protzek}

	Introdução a Ruby on Rails voltado para designers, apresentando
	a linguagem, framework e diversas ferramentas populares.

	{\em\scriptsize Tecnologias: Ruby on Rails.}

\item[2012] Microcontrolador MSP430 com memória FRAM\\
	{\em\small TECHtraininG}

	Apresentação das principais características da microcontroladora MSP430
	da \emph{Texas Instruments} e sua memória ferroelétrica FRAM.

\item[2011] Arduino Hack Day\\
	{\em\small Mecajun--IEEE}

	Introdução a eletrônica e programação utilizando a microcontroladora
	Arduino. Ministrado pelo inventor \emph{Lucas Fragomeni}.

	{\em\scriptsize Tecnologias: Arduino, C.}

\end{topic}

\section{Projetos}

\begin{topic}
%\item[2017--Agora] Site da \emph{APROSMIG}
%
%	Encarregado técnico pelo site da \emph{APROSMIG} -- Associação das
%	Prostitutas de Minas Gerais.
%
%	{\scriptsize\url{http://aprosmig.org.br}}
%
%	{\em\scriptsize Tecnologias: Ruby, Jekyll, Github Pages.}

\item[2017--Agora] Introdução da Raspberry PI à comunidade do \emph{Igapó-Açu}

	Estou escrevendo um material de introdução à computação e ao hardware
	da Raspberry PI voltado para jovens da comunidade ribeirinha do
	\emph{Igapó-Açu}, Amazonas, Brasil.

	O propósito do material é ser um guia prático para a utilização e
	abordar impactos da tecnologia na sociedade.

	{\em\scriptsize Tecnologias: Raspberry PI, \LaTeX{}.}

\item[2006--Agora] Motor de jogos 3D

	Meu próprio motor de jogos 3D, que utiliza uma árvore BSP para acelerar
	gráficos e testes de colisão. Inspirado em \emph{Quake}, escrito do
	zero em C, com libSDL e OpenGL.

	{\scriptsize\url{https://github.com/ayghor/teh_engine}}\\
	{\scriptsize\url{https://github.com/ayghor/engine-on-Qua-Ago-29-09\_42\_48-BRT-2007}}

	{\em\scriptsize Tecnologias: C, OpenGL, OpenGL ES 2.0, libSDL, Python,
	Blender3D.}

%\item[2010--2011] Pesquisa sobre métricas de vídeo sem referência\\
%	{\em\small UnB--Mylène Farias}
%
%	Breve pesquisa sobre métricas de vídeo sem referência, coordenada pela
%	professora \emph{Mylène Farias}. Implementei um algorítmo para detectar
%	descontinuidades temporais em vídeos.
%
%	{\em\scriptsize Tecnologias: C, Gnuplot.}

\item[2006--2011] Linux From Scratch

	Minha própria distribuição Linux baseada em Slackware e no livro Linux
	From Scratch. Desenvolvida e utilizada por mais de 4 anos, até a morte
	de seu computador hospedeiro.

	Destaques:

	\begin{enumerate}
		\item framework shellscript para compilar os pacotes de
			software

		\item aproximadamente 300 pacotes de software

		\item gerenciador de pacotes binários

		\item x86\_64 com suporte multilib de 64 e 32bits

		\item scripts de inicialização estilo BSD

		\item máxima conformidade possível com o FHS (Filesystem
			Hierarchy Standard)
	\end{enumerate}

	{\scriptsize\url{https://github.com/ayghor/rogix\_gordax}}

	{\em\scriptsize Tecnologias: sh, C.}

\end{topic}

\section{Tecnologias}

\begin{multicols}{3}
	\raggedcolumns
	\begin{itemize}
		\item Linux
		\item C
		\item Ruby
		\item Java
		\item Vim
		\item Git
		\item HTML
		\item JavaScript
		\item jQuery
		\item Ruby on Rails
		\item Jekyll
		\item PostgreSQL
		\item AWS
		\item Heroku
		\item OpenGL
		\item \LaTeX{}
		\item Bitcoin
		\item Octave
		\item Machine Learning
	\end{itemize}
\end{multicols}

\section{Idiomas}

\begin{multicols}{3}
	\raggedcolumns
	\begin{itemize}
		\item Português
		\item Inglês
	\end{itemize}
\end{multicols}

%\section{Certificados}
%
%\begin{thebibliography}{1}
%\bibitem{mlcert} {}
%\end{thebibliography}

\end{document}
