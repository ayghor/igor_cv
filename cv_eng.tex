\documentclass[a4paper]{simplecv}
\usepackage[utf8]{inputenc}
%\usepackage[brazilian]{babel}
\usepackage[hidelinks]{hyperref}
\usepackage{multicol}

\begin{document}

\leftheader{Avenida Central Bloco 1645\\
Lote 10, ap 102\\
Núcleo Bandeirante\\
Brasília, DF\\
71710-560 \\
Brasil}

\rightheader{\href{tel:+5561999894025}{+55 61 99989-4025}\\
\href{mailto:rogi@skylittlesystem.org}{rogi@skylittlesystem.org}\\
\url{https://github.com/ayghor}}

\title{Igor de Sant'Ana Fontana}

\maketitle

I'm a self-taught programmer and math student who loves elegant solutions to
real problems. I write code in \emph{C} since age 12 ($\approx 17$ years ago)
and have been working with \emph{Ruby} and \emph{Rails} for the last 5 years.

I believe that technology should be simple and useful. I like working with
\emph{Linux} and \emph{Free Software} in general. Also have interest in
\emph{Videogames}, \emph{Machine Learning}, teaching and playing guitar.

\section{Experience}

\begin{topic}
\item[2017] Ruby Backend Developer\\
	{\em\small Wololo--We Convert}

	In charge of modeling and implementing the backend of an advertising
	and militant management platform. Worked in parallel with frontend
	developers and designers.

	Modeled the database and the business logic of the application to match
	sketches made by product designers. Implemented a JSON API using
	minimalist libraries to ensure the service was fast and lightweight.
	Wrote detailed documentation for the web, iOS and Android developers.
	Set up test and production servers.

	{\em\scriptsize Technologies: Ruby, Sequel (ruby gem), Roda (ruby gem),
	PostgreSQL, AWS (Amazon Web Services), EC2 (Elastic Compute Cloud), SES
	(Simple Email Service), SNS (Simple Notification Service), FCM
	(Firebase Cloud Messaging), Facebook Graph API, Facebook Login.}

\item[2016--2017] Ruby on Rails Fullstack Developer\\
	{\em\small Papo de Homem}

	Responsible for solving issues and implementing new features both in
	the backend and in the frontend of the \emph{Papo de Homem} blog.

	Fixed many bugs. Migrated legacy database. Implemented SSL. Created a
	JSON API and implemented Push notifications for integration with the
	mobile apps.

	{\em\scriptsize \url{https://papodehomem.com.br}}

	{\em\scriptsize Technologies: Ruby on Rails, HTML, Slim (ruby gem),
	JavaScript, LetsEncrypt, Heroku, PostgreSQL, FCM (Firebase Cloud
	Messaging), Elasticsearch, Redis, SideKiq, Facebook Social Plugins.}

\item[2013--2014] Rails and Java Fullstack Developer\\
	{\em\small LambLamb--Impressões Impressionantes}

	Responsible for maintaining, developing, operating and supporting other
	operators of the Instagram photo printer -- \emph{LambLamb}. Worked
	along another programmer.

	Improved the print quality by replacing the rasterized image format
	with a vectorized one. Created a WYSIWYG print template editor, for
	customization with client logos and media metadata. Operated and
	introduced \emph{LambLamb} to the public during events.

	{\em\scriptsize Technologies: Ruby on Rails, Java, Java Printing API,
	Java 2D Graphics API, Swing, HTML, JavaScript, Instagram API.}

\item[2011--2013] Júnior Developer\\
	{\em\small Bebop--Computação Criativa}

	Assistant in the development of interactive objects and installations.

	Improved the performance, specially graphic, of several of the company's
	projects. Mathematically modeled physical problems. Operated and
	presented the creations to the public during events.

	{\em\scriptsize Technologies: C, Java, Processing, Arduino, XBee,
	Kinect, OpenCV, OpenGL.}
\end{topic}

\section{Academic Formation}

\begin{topic}
\item[2009--2019] Bachelor of Mathematics\\
	{\em\small UnB--Universidade de Brasília}

\end{topic}

\section{Certificates}

\begin{topic}
\item[2016] Machine Learning\\
	{\em\small Coursera--Stanford}

	Broad introduction to machine learning, data mining and statistical
	pattern recognition.

	Topics include:
	\begin{enumerate}
		\item Supervised learning (parametric/non-parametric
			algorithms, support vector machines, kernels, neural
			networks)

		\item Unsupervised learning (clustering, dimensionality
			reduction, recommender systems, deep learning)

		\item Best practices in machine learning (bias/variance theory;
			innovation process in machine learning and AI)
	\end{enumerate}

	Taught by professor \emph{Andrew Ng} from \emph{Stanford University}
	through the online platform \emph{Coursera}.

	{\scriptsize\url{https://www.coursera.org/account/accomplishments/verify/NE32SCRBV3F6}}

	{\em\scriptsize Technologies: Octave.}
\end{topic}

\section{Courses}

\begin{topic}
\item[2017] Bitcoin and Cryptocurrency Technologies\\
	{\em\small Coursera--Princeton}

	Introduction to the technical, economic and social aspects of the
	Bitcoin cryptographic currency.

	Topics include:
	\begin{enumerate}
		\item Criptography (symmetric and asymmetric, signature
			digital, hash, hash pointer, block chain, merkle tree)
		\item Bitcoin decentralization (byzantine generals' problem)
		\item Community, politics and regulation
		\item Bitcoin as a platform
		\item Altcoins
	\end{enumerate}

	Taught by professor \emph{Arvind Narayanan} of \emph{Princeton
	University} through the online platform \emph{Coursera}.

	{\em\scriptsize Technologies: Java.}

%\item[2013] Rails para Designers\\
%	{\em\small Julio Protzek}
%
%	Introduction to Ruby on Rails aimed at designers, presenting the
%	language, framework and various popular tools.
%
%	{\em\scriptsize Technologies: Ruby on Rails.}

\item[2012] Microcontrolador MSP430 com memória FRAM\\
	{\em\small TECHtraininG}

	Presentation of the main features of the Texas Instruments' MSP430
	microcontroller and its FRAM ferroelectric memory.

\item[2011] Arduino Hack Day\\
	{\em\small Mecajun--IEEE}

	Introduction to electronics and programming using the Arduino
	microcontroller. Taught by inventor \emph{Lucas Fragomeni}.

	{\em\scriptsize Technologies: Arduino, C.}

\end{topic}

\section{Projects}

\begin{topic}
%\item[2017--Now] \emph{APROSMIG} website
%
%	In charge of the technical aspects of the \emph{APROSMIG} -- Associação
%	das Prostitutas de Minas Gerais (Minas Gerais' Sex Workers Association)
%	-- website.
%
%	{\scriptsize\url{http://aprosmig.org.br}}
%
%	{\em\scriptsize Technologies: Ruby, Jekyll, Github Pages.}

\item[2017--Now] Introduction of the Raspberry PI to the \emph{Igapó-Açu} community

	I'm writing an introduction to computing and the Raspberry PI hardware
	aimed at the youth of the \emph{Igapó-Açu} riverside community,
	Amazonas, Brasil.

	The purpose of the material is to be a practical guide to using
	technology and address some of it's impacts on society.

	{\em\scriptsize Technologies: Raspberry PI, \LaTeX{}.}

\item[2006--Now] 3D Game Engine

	My own 3D game engine, which uses a BSP tree to accelerate graphics and
	collision tests. Inspired in \emph{Quake}, written from scratch in C,
	with libSDL and OpenGL.

	{\scriptsize\url{https://github.com/ayghor/teh_engine}}\\
	{\scriptsize\url{https://github.com/ayghor/engine-on-Qua-Ago-29-09\_42\_48-BRT-2007}}

	{\em\scriptsize Technologies: C, OpenGL, OpenGL ES 2.0, libSDL, Python,
	Blender3D.}

%\item[2010--2011] Research on no-reference video metrics\\
%	{\em\small UnB--Mylène Farias}
%
%	Brief research on no-reference video metrics, under professor
%	\emph{Mylène Farias}. Implemented an algorithm for detecting temporal
%	discontinuities in videos.
%
%	{\em\scriptsize Technologies: C, Gnuplot.}

\item[2006--2011] Linux From Scratch

	My own Linux distro based on Slackware and the Linux From Scratch book.
	Developed and used for more than 4 years, until it's host computer
	died.

	Features:

	\begin{enumerate}
		\item shellscript framework for building software packages

		\item around 300 software packages

		\item binary package manager

		\item x86\_64 with 64 and 32bit multilib support

		\item BSD-style init scripts

		\item maximum possible conformance with the FHS (Filesystem
			Hierarchy Standard)
	\end{enumerate}

	{\scriptsize\url{https://github.com/ayghor/rogix\_gordax}}

	{\em\scriptsize Technologies: sh, C.}

\end{topic}

\section{Technologies}

\begin{multicols}{3}
	\raggedcolumns
	\begin{itemize}
		\item Linux
		\item C
		\item Ruby
		\item Java
		\item Vim
		\item Git
		\item HTML
		\item JavaScript
		\item jQuery
		\item Ruby on Rails
		\item Jekyll
		\item PostgreSQL
		\item AWS
		\item Heroku
		\item OpenGL
		\item \LaTeX{}
		\item Bitcoin
		\item Octave
		\item Machine Learning
	\end{itemize}
\end{multicols}

\section{Languages}

\begin{multicols}{3}
	\raggedcolumns
	\begin{itemize}
		\item Portuguese
		\item English
	\end{itemize}
\end{multicols}

%\section{Certificados}
%
%\begin{thebibliography}{1}
%\bibitem{mlcert} {}
%\end{thebibliography}

\end{document}
